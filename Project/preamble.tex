% Common preamble I use
\usepackage[utf8]{inputenc}         % Input encoding
\usepackage[T1]{fontenc}            % Font encoding
\usepackage{textcomp}               % Provides various additional symbols and text-related features
\usepackage{microtype}              % Improves micro-typography of the document
\usepackage{pifont}                 % Provides additional symbols, including dingbats
\usepackage{changepage}             % Allows adjustment of page layout parameters
\usepackage{ftnxtra}
\usepackage{fancyhdr}

\usepackage{chemgreek,textgreek}    % Greek letters 

\usepackage{authblk}                % For customizing author block layout in the title page

\usepackage{listings}               % Code listings
\usepackage{xcolor}                 % Color support

\usepackage{graphicx}               % Graphics support
\usepackage[top=2.5cm,bottom=2.5cm,left=2.5cm,right=2.5cm]{geometry}
\usepackage{tikz}                   % For in-doc drawings
\usepackage{subcaption}             % Enhanced support for sub-figures and sub-captions
\usepackage{amsmath}                % Math support
%\usepackage{mathenv}                % Provides additional math environments
\usepackage{amssymb}                % Provides various additional mathematical symbols
\usepackage{amsthm}                 % Provides enhanced support for theorem-like environments
\usepackage{wrapfig}
\usepackage{pgfplots}
\usepackage{amsmath}


\usepackage{xtab}                   % Tables with adjustable width 
\usepackage{multirow}               % For multi-row cells in tables
\usepackage{array}                  % Provides more flexible and customizable array and tabular environments
\usepackage{longtable}              % Allows tables that span multiple pages
\usepackage{threeparttablex}
\usepackage{tabularray}

\usepackage{booktabs}               % Enhanced tables
\usepackage{csquotes}               % Quotation marks
\usepackage{algorithm}
\usepackage{siunitx}
\sisetup{
  input-ignore={,},
  input-decimal-markers={.},
  group-separator={,},
  table-align-text-pre=false,
  table-align-text-post=true
}
%\usepackage{algorithmic}            % For typesetting algorithms
\usepackage{algpseudocode}
\usepackage{caption}                % Customizes captions in floating environments
\usepackage{subcaption}
\usepackage{setspace}
\DeclareMathOperator{\atan}{atan}
\DeclareCaptionLabelSeparator*{spaced}{\\[2ex]}
\captionsetup[table]{textfont=bf,format=plain,justification=justified, singlelinecheck=false,labelsep=newline,skip=0pt}
\DeclareCaptionLabelFormat{apafigure}{\textbf{Figure #2}}
\DeclareCaptionLabelFormat{procedure}{\textbf{Procedure #2}}
\DeclareCaptionLabelFormat{apalisting}{\textbf{Listing #2}}
\captionsetup[figure]{labelformat=apafigure, labelsep=newline, justification=justified, singlelinecheck=false}


\usepackage{ragged2e}               % Enhanced text alignment commands

\usepackage{enumitem}               % Customizable lists     

\usepackage[sort&compress]{natbib}  % Cite style
\setcitestyle{numbers,square,comma}

\usepackage{times}                  % Times font       

\usepackage{hyperref}               % Hyperlinks
\usepackage{url}                    % For typesetting URLs with line breaks at hyphens
\def\UrlBreaks{\do\/\do-}
\hypersetup{breaklinks=true}
\urlstyle{same}
\DeclareUrlCommand\email{}          % Email command
\usepackage{nameref}                % Enables referencing section names instead of numbers

% For using TODO notes
% \todo[inline,caption={}]{TODOs are to be inserted like this}
% \usepackage[color=blue!10,textsize=footnotesize,textwidth=25mm]{todonotes}
\usepackage[disable]{todonotes}
\usepackage{float}


\makeatletter
% Commands to format a counter value as Greek letter to be used like 
% \arabic or \roman:
\newcommand*\alphgreek[1]{\expandafter\@alphgreek\csname c@#1\endcsname}
\newcommand*\@alphgreek[1]{\csname chemgreekIntToGreek:n\endcsname{#1}}
\newcommand*\Alphgreek[1]{\expandafter\@Alphgreek\csname c@#1\endcsname}
\newcommand*\@Alphgreek[1]{\csname chemgreekIntToGreek:n\endcsname{#1}}

% Register new counter formats to enumitem:
\AddEnumerateCounter*{\alphgreek}{\@alphgreek}{\chemalpha}
\AddEnumerateCounter*{\Alphgreek}{\@Alphgreek}{\chemAlpha}
\makeatother

% \code{} command to insert in-text code snippets
\definecolor{light-gray}{gray}{0.95}
\newcommand{\code}[1]{\colorbox{light-gray}{\texttt{#1}}}

% AUTHOR
\newcommand{\authorname}{Samuel Jonsson}
\newcommand{\authormail}{\href{mailto:sajs19@student.bth.se}{sajs19@student.bth.se}}
\newcommand{\authorpersonnr}{19990415-5596}
\newcommand{\program}{DVAMI19h}


% REPORT METADATA
\newcommand{\doctitle}{Title}
\newcommand{\docdate}{Latest version: \today}

\pagestyle{fancy}     % Use the fancy page style
\fancyhf{}            % Clear all headers and footers

% HEADER CUSTOMISATION
\fancyhead[L]{\textit{\doctitle}}
\fancyhead[R]{\textit{\docdate}}

% FOOTER CUSTOMISATION
\fancyfoot[L]{\authorname~--~\program} % Left side: author name and program
\fancyfoot[R]{Page \thepage}
