\documentclass[12pt,a4paper,twoside]{article}

% Common preamble I use
\usepackage[utf8]{inputenc}         % Input encoding
\usepackage[T1]{fontenc}            % Font encoding
\usepackage{textcomp}               % Provides various additional symbols and text-related features
\usepackage{microtype}              % Improves micro-typography of the document
\usepackage{pifont}                 % Provides additional symbols, including dingbats
\usepackage{changepage}             % Allows adjustment of page layout parameters
\usepackage{ftnxtra}
\usepackage{fancyhdr}

\usepackage{chemgreek,textgreek}    % Greek letters 

\usepackage{authblk}                % For customizing author block layout in the title page

\usepackage{listings}               % Code listings
\usepackage{xcolor}                 % Color support

\usepackage{graphicx}               % Graphics support
\usepackage[top=2.5cm,bottom=2.5cm,left=2.5cm,right=2.5cm]{geometry}
\usepackage{tikz}                   % For in-doc drawings
\usepackage{subcaption}             % Enhanced support for sub-figures and sub-captions
\usepackage{amsmath}                % Math support
%\usepackage{mathenv}                % Provides additional math environments
\usepackage{amssymb}                % Provides various additional mathematical symbols
\usepackage{amsthm}                 % Provides enhanced support for theorem-like environments
\usepackage{wrapfig}
\usepackage{pgfplots}
\usepackage{amsmath}


\usepackage{xtab}                   % Tables with adjustable width 
\usepackage{multirow}               % For multi-row cells in tables
\usepackage{array}                  % Provides more flexible and customizable array and tabular environments
\usepackage{longtable}              % Allows tables that span multiple pages
\usepackage{threeparttablex}
\usepackage{tabularray}

\usepackage{booktabs}               % Enhanced tables
\usepackage{csquotes}               % Quotation marks
\usepackage{algorithm}
\usepackage{siunitx}
\sisetup{
  input-ignore={,},
  input-decimal-markers={.},
  group-separator={,},
  table-align-text-pre=false,
  table-align-text-post=true
}
%\usepackage{algorithmic}            % For typesetting algorithms
\usepackage{algpseudocode}
\usepackage{caption}                % Customizes captions in floating environments
\usepackage{subcaption}
\usepackage{setspace}
\DeclareMathOperator{\atan}{atan}
\DeclareCaptionLabelSeparator*{spaced}{\\[2ex]}
\captionsetup[table]{textfont=bf,format=plain,justification=justified, singlelinecheck=false,labelsep=newline,skip=0pt}
\DeclareCaptionLabelFormat{apafigure}{\textbf{Figure #2}}
\DeclareCaptionLabelFormat{procedure}{\textbf{Procedure #2}}
\DeclareCaptionLabelFormat{apalisting}{\textbf{Listing #2}}
\captionsetup[figure]{labelformat=apafigure, labelsep=newline, justification=justified, singlelinecheck=false}


\usepackage{ragged2e}               % Enhanced text alignment commands

\usepackage{enumitem}               % Customizable lists     

\usepackage[sort&compress]{natbib}  % Cite style
\setcitestyle{numbers,square,comma}

\usepackage{times}                  % Times font       

\usepackage{hyperref}               % Hyperlinks
\usepackage{url}                    % For typesetting URLs with line breaks at hyphens
\def\UrlBreaks{\do\/\do-}
\hypersetup{breaklinks=true}
\urlstyle{same}
\DeclareUrlCommand\email{}          % Email command
\usepackage{nameref}                % Enables referencing section names instead of numbers

% For using TODO notes
% \todo[inline,caption={}]{TODOs are to be inserted like this}
% \usepackage[color=blue!10,textsize=footnotesize,textwidth=25mm]{todonotes}
\usepackage[disable]{todonotes}
\usepackage{float}


\makeatletter
% Commands to format a counter value as Greek letter to be used like 
% \arabic or \roman:
\newcommand*\alphgreek[1]{\expandafter\@alphgreek\csname c@#1\endcsname}
\newcommand*\@alphgreek[1]{\csname chemgreekIntToGreek:n\endcsname{#1}}
\newcommand*\Alphgreek[1]{\expandafter\@Alphgreek\csname c@#1\endcsname}
\newcommand*\@Alphgreek[1]{\csname chemgreekIntToGreek:n\endcsname{#1}}

% Register new counter formats to enumitem:
\AddEnumerateCounter*{\alphgreek}{\@alphgreek}{\chemalpha}
\AddEnumerateCounter*{\Alphgreek}{\@Alphgreek}{\chemAlpha}
\makeatother

% \code{} command to insert in-text code snippets
\definecolor{light-gray}{gray}{0.95}
\newcommand{\code}[1]{\colorbox{light-gray}{\texttt{#1}}}

% AUTHOR
\newcommand{\authorname}{Samuel Jonsson}
\newcommand{\authormail}{\href{mailto:sajs19@student.bth.se}{sajs19@student.bth.se}}
\newcommand{\authorpersonnr}{19990415-5596}
\newcommand{\program}{DVAMI19h}


% REPORT METADATA
\newcommand{\doctitle}{Title}
\newcommand{\docdate}{Latest version: \today}

\pagestyle{fancy}     % Use the fancy page style
\fancyhf{}            % Clear all headers and footers

% HEADER CUSTOMISATION
\fancyhead[L]{\textit{\doctitle}}
\fancyhead[R]{\textit{\docdate}}

% FOOTER CUSTOMISATION
\fancyfoot[L]{\authorname~--~\program} % Left side: author name and program
\fancyfoot[R]{Page \thepage}


\setlist{topsep=1ex,itemsep=0.5ex,parsep=0pt,partopsep=0pt}

\begin{document}

\vspace*{3cm}

\begin{center}
    \Huge\textbf{\doctitle}\\
    \Large\textbf{\docsubtitle}
    \\\vspace*{5mm} 
    \large\today                            
\end{center} 

\vspace*{5mm}

\begin{tabular}{|l|p{13cm}|}
    \hline
    \textbf{Name}           & \authorname     \\
    \hline
    \textbf{E-Mail}         & \authormail     \\
    \hline
    \textbf{Person Nr.}     & \authorpersonnr \\
    \hline
    \textbf{Program}        & \program        \\
    \hline
\end{tabular}

\begin{figure}[!b]
    \centering
    \includegraphics[width = 0.25\textwidth]{figures/BTH_logo_black.png}
\end{figure}

\newpage

\section{Setup}
\label{sec:setup}
\todo[inline,caption={}]{
    \begin{itemize}
        \item Describe the data and the analysis problem.
        \item Choose and describe the modeling approach (e.g., non-hierarchical or hierarchical model).
        \item Justify your prior choice.
        \item Perform posterior predictive checks.
    \end{itemize}
}
\subsection{Analysis problem}
\label{ssec:analysisproblem}

\subsection{Data Selection}
\label{ssec:dataselection}
\todo[inline,caption={}]{Describe the data and the analysis problem.}

The dataset selected is a datasets containing a list of emails, as well as a label marking each
email as "spam" or "ham" (spam or not spam). The first 10 rows of the dataset looks as follows:

\begin{longtable}[!ht]{p{1.5cm}|p{11cm}}
    \caption{\code{mail\_data.csv} dataset first 10 rows}\\
    \textbf{Category} & \textbf{Message} \\
    \hline
    ham      & "Go until jurong point, crazy.. Available only in bugis n great world la e buffet...
               Cine there got amore wat..." \\
    \hline
    ham      & Ok lar... Joking wif u oni... \\
    \hline
    spam     & Free entry in 2 a wkly comp to win FA Cup final tkts 21st May 2005. Text FA to 87121 to
               receive entry question(std txt rate)T\&C's apply 08452810075over18's \\
    \hline
    ham      & U dun say so early hor... U c already then say... \\
    \hline
    ham      & "Nah I don't think he goes to usf, he lives around here though" \\
    \hline
    spam     & "FreeMsg Hey there darling it's been 3 week's now and no word back! I'd like some fun you
                up for it still? Tb ok! XxX std chgs to send, \pounds 1.50 to rcv" \\
    \hline
    ham      & Even my brother is not like to speak with me. They treat me like aids patent. \\
    \hline
    ham      & As per your request 'Melle Melle (Oru Minnaminunginte Nurungu Vettam)' has
               been set as your callertune for all Callers. Press *9 to copy your friends Callertune \\
    \hline
    ...      & ... \\
\end{longtable}

Then, using a Python script, the labels were converted to 1 if it was "spam" and 0 if it was "ham",
for easier analysis.

\subsection{Model}
\label{ssec:model}
\todo[inline,caption={}]{
    \begin{itemize}
        \item Choose and describe the modeling approach (e.g., non-hierarchical or hierarchical model).
        \item Justify your prior choice.
    \end{itemize}
}
The model chosen was a binomial likelihood model with a beta prior. As the goal is to analyse the
probability of an email being spam, the fallout will be binary (either it is spam or it is not). Hence,
a binomial likelihood, where I want to find the parameter $\theta$ in a dataset of fixed size with a set
number of "successes" and "fails" (spam and ham), is appropriate.

Additionally, as I do not have any prior knowledge in regards to this distribution, a non-informative prior
is the most suited option, and as $Beta(1,1)$ is a common prior used with binomial likelihood functions, I
chose it for this problem.

\subsection{Prior checks}
\label{ssec:priorchecks}
\todo[inline,captio?{}]{Perform posterior predictive checks.}

\section{Results}
\todo[inline,caption={}]{Include diagnostics to assess model convergence and adequacy.}

\section{Discussion}
\todo[inline,caption={}]{Discuss results, problems encountered, and possible improvements.}

\appendix
\section{R Code}
\lstinputlisting[language=R,style=Rstyle,caption=Project R code,label=apx:RCode,]{./R/base.r}
\end{document}