% Common preamble I use
\usepackage[utf8]{inputenc}         % Input encoding
\usepackage[T1]{fontenc}            % Font encoding
\usepackage{textcomp}               % Provides various additional symbols and text-related features
\usepackage{microtype}              % Improves micro-typography of the document
\usepackage{pifont}                 % Provides additional symbols, including dingbats
\usepackage{changepage}             % Allows adjustment of page layout parameters
\usepackage{ftnxtra}                % Allows for fancy footnotes
\usepackage{fancyhdr}               % Allows for headers and footers
\usepackage{subcaption}             % Enhanced support for sub-figures and sub-captions
\usepackage{wrapfig}

\usepackage{chemgreek,textgreek}    % Greek letters 

\usepackage{authblk}                % For customizing author block layout in the title page

% Listings setup
\usepackage{listings}               % Code listings
\usepackage{xcolor}                 % Color support
% Define R language for listings
\lstdefinelanguage{R}{
    keywords={function,
              if,
              else,
              repeat,
              while,
              for,
              in,
              next,
              break,
              TRUE,
              FALSE,
              NULL,
              Inf,
              NaN,
              NA,
              NA_integer_,
              NA_real_,
              NA_complex_,
              NA_character_,
              library,
              require,
              install.packages},
    keywordstyle=\color{blue}\bfseries,
    sensitive=true,
    comment=[l]{\#},
    commentstyle=\color{green!50!black},
    stringstyle=\color{orange},
    morestring=[b]",
    morestring=[b]',
    moredelim=[is][\color{purple}]{`}{`}
}
\lstdefinestyle{Rstyle}{
    language=R,                
    basicstyle=\ttfamily\small,
    keywordstyle=\color{blue}, 
    stringstyle=\color{orange},
    commentstyle=\color{green!50!black}, 
    backgroundcolor=\color{gray!10},    
    showstringspaces=false,    
    numbers=left,              
    numberstyle=\tiny,         
    breaklines=true,           
    frame=single,              
}

% Graphic packages
\usepackage{graphicx}               % Graphics support
\usepackage[top=2.5cm,bottom=2.5cm,left=2.5cm,right=2.5cm]{geometry}
\usepackage{tikz}                   % For in-doc drawings
\usepackage{pgfplots}

% Maths packages
\usepackage{amsmath}                % Math support
\usepackage{amssymb}                % Provides various additional mathematical symbols
\usepackage{amsthm}                 % Provides enhanced support for theorem-like environments

% Table support packages
\usepackage{xtab}                   % Tables with adjustable width 
\usepackage{multirow}               % For multi-row cells in tables
\usepackage{array}                  % Provides more flexible and customizable array and tabular environments
\usepackage{longtable}              % Allows tables that span multiple pages
\usepackage{threeparttablex}        % Helps avoid clashes between tables and footnotes
\usepackage{tabularray}             % Combines features of longtable, array, multirow, etc.

\usepackage{booktabs}               % Enhanced tables
\usepackage{csquotes}               % Quotation marks and quote formatting
\usepackage{siunitx}                % For typesetting numbers and units.
\sisetup{                           % Centers numbers in tables by the decimal point.
  input-ignore={,},
  input-decimal-markers={.},
  group-separator={,},
  table-align-text-pre=false,
  table-align-text-post=true
  }

\usepackage{algorithm}              % Allows insertion of algorithms as an object
\usepackage{algpseudocode}          % Used to insert pseudocode  
\usepackage{caption}                % Customizes captions in floating environments

\usepackage{setspace}

\usepackage{ragged2e}               % Enhanced text alignment commands

\usepackage{enumitem}               % Customizable lists     

\usepackage[sort&compress]{natbib}  % Cite style
\setcitestyle{numbers,square,comma}

\usepackage{times}                  % Times font       

\usepackage{hyperref}               % Hyperlinks
\usepackage{url}                    % For typesetting URLs with line breaks at hyphens
\def\UrlBreaks{\do\/\do-}           % Allows urls to break lines
\hypersetup{breaklinks=true}
\urlstyle{same}
\usepackage{nameref}                % Enables referencing section names instead of numbers

% For using TODO notes
% \todo[inline,caption={}]{TODOs are to be inserted like this}
\usepackage[color=blue!10,textsize=footnotesize,textwidth=25mm]{todonotes}
%\usepackage[disable]{todonotes}
\usepackage{float}

% CUSTOM LABELS
\DeclareCaptionLabelFormat{apafigure}{\textbf{Figure #2}}
\DeclareCaptionLabelFormat{procedure}{\textbf{Procedure #2}}
\DeclareCaptionLabelFormat{apalisting}{\textbf{Listing #2}}

% CUSTOM CAPTION FORMATS
%\captionsetup[figure]{labelformat=apafigure,
%                      labelsep=newline,
%                      justification=justified,
%                      singlelinecheck=false}
%\DeclareMathOperator{\atan}{atan}
%\captionsetup[lstlisting]{labelformat=apalisting,
%                          position=above,
%                          justification=raggedright,
%                          labelsep=newline,
%                          singlelinecheck=false,}
%\DeclareCaptionLabelSeparator*{spaced}{\\[2ex]}
%\captionsetup[table]{textfont=bf,
%                     format=plain,
%                     justification=justified,
%                     singlelinecheck=false,
%                     labelsep=newline,
%                     skip=0pt}
\makeatletter
% Commands to format a counter value as Greek letter to be used like 
% \arabic or \roman when using enumerate:
\newcommand*\alphgreek[1]{\expandafter\@alphgreek\csname c@#1\endcsname}
\newcommand*\@alphgreek[1]{\csname chemgreekIntToGreek:n\endcsname{#1}}
\newcommand*\Alphgreek[1]{\expandafter\@Alphgreek\csname c@#1\endcsname}
\newcommand*\@Alphgreek[1]{\csname chemgreekIntToGreek:n\endcsname{#1}}

% Register new counter formats to enumitem:
\AddEnumerateCounter*{\alphgreek}{\@alphgreek}{\chemalpha}
\AddEnumerateCounter*{\Alphgreek}{\@Alphgreek}{\chemAlpha}
\makeatother

% \code{} command to insert in-text code snippets
\definecolor{light-gray}{gray}{0.95}
\newcommand{\code}[1]{\colorbox{light-gray}{\texttt{#1}}}

% AUTHOR
% N/A

% REPORT METADATA
\newcommand{\doctitle}{MS2505: Bayesian Statistics}
\newcommand{\docsubtitle}{Compiled exercises}
\newcommand{\docdate}{Latest version: \today}

\pagestyle{fancy}                                 % Use the fancy page style
\fancyhf{}                                        % Clear all headers and footers

% HEADER CUSTOMISATION
\fancyhead[L]{\textit{\doctitle~--~\docsubtitle}} % Header float left
\fancyhead[R]{\textit{\docdate}}                  % Header float right

% FOOTER CUSTOMISATION
\fancyfoot[R]{Page \thepage}                      % Footer float right
